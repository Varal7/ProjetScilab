\documentclass[11pt,a4paper,french]{article}

\usepackage[utf8]{inputenc}
\usepackage[T1]{fontenc}

\usepackage[french]{babel}

\usepackage{amssymb,amsmath, bm}
\title{Retour vers le futur}
\author{Qi \textsc{Sun}}
\date{22 juin 2015}%date{} pour date vide

\def \P{\mathbb{P}}
\def \E{\mathcal{E}}
\def \esp{\mathbb{E}}

\begin{document}

\maketitle

\tableofcontents
\clearpage

\section{Coalescence et loi invariante}

\paragraph{T2.}

\[\mathbb{P}(A_1)=\mathbb{P}(\{F_{0,{n_0}k}(X)=F_{0,{n_0}k}(Y)\})\]
On en déduit que:
\[ \mathbb{P}(A_1)=\sum_{x,y} \mathbb{P}(F_{0,{n_0}k}(x)=F_{0,{n_0}k}(y))\mathbb{P}(X=x,Y=y)\]
Comme on a supposé que 
\[\inf_{x,y}\mathbb{P}(F_{0,{n_0}k}(x)=F_{0,{n_0}k}(y))\geq\varepsilon>0,\]
\[ \mathbb{P}(A_1)\geq\sum_{x,y} \varepsilon\mathbb{P}(X=x,Y=y)\geq\varepsilon\sum_{x,y}\mathbb{P}(X=x,Y=y)\geq\varepsilon\]
Ainsi,
\[ \mathbb{P}(A_1)\geq\varepsilon\]
\\
Montrons par récurrence sur $l$ que:
\[ \mathbb{P}(S>l)=\mathbb{P}({A_1}^c,{A_2}^c,...,{A_l}^c)\le(1-\varepsilon)^l. \]
Le résultat au rang $1$ se déduit du point précédent.
Soit $l\geq1$, supposons le résultat au rang $l$. Alors
\begin{equation*}
\begin{split}
\mathbb{P}(S>l+1)&=\mathbb{P}({A_1}^c,{A_2}^c,...,{A_l}^c,{A_{l+1}}^c)\\
&=\sum_{x,y} \mathbb{P}(F_{0,n_0}(x)\neq F_{0,n_0}(y),...,F_{0,{n_0}(l+1)}(x)\neq F_{0,{n_0}(l+1)}(y))\mathbb{P}(X=x,Y=y)\\
%\begin{multline*}
&=\sum_{x,y} \mathbb{P}(F_{n_0l,n_0(l+1)}(F_{0,{n_0}l}(x))\neq F_{n_0l,n_0(l+1)}(F_{0,{n_0}l}(y))) \\
&\hspace{1cm}\times\mathbb{P}(F_{0,n_0}(x)\neq F_{0,n_0}(y),...,F_{0,{n_0}l}(x)\neq F_{0,{n_0}l}(y))\mathbb{P}(X=x,Y=y)\\
%\end{multline*}
&\leq (1-\varepsilon)\mathbb{P}(S>l)\\
&\leq (1-\varepsilon)^{(l+1)}
\end{split}
\end{equation*}
D'où le résultat par récurrence.
On en déduit:
\[\mathbb{E}[S]=\sum_{l\geq0}\mathbb{P}(S>l)\leq\sum_{l\geq0}(1-\varepsilon)^l\]
Donc on trouve:
\[\mathbb{E}[S]\leq\frac{1}{\varepsilon}\]
\\
Comme $(\mathbb{P}(F_{0,n}(X)=F_{0,n}(Y)))_n\in \mathbb{N}$ est une suite croissante sur $n$.\\
Or, on a pour tout $n\geq 1$, $\mathbb{P}(F_{0,nn_0}(X)=F_{0,nn_0}(Y))\geq\mathbb{P}(S\leq n)$.\\
Ainsi,
\[1-(1-\epsilon)^n\leq\mathbb{P}(F_{0,nn_0}(X)=F_{0,nn_0}(Y))\leq1\]
D'après le théorème d'encadrement et la monotonie de la suite, on a:
\[\lim\limits_{n \rightarrow +\infty} \mathbb{P}(F_{0,n}(X)=F_{0,n}(Y)) = 1\]

\paragraph{T3.}
Supposons que $\pi$ et $\nu$ sont deux probabilités invariantes. 
Alors pour tout $n\geq0$, $F_{0,n}(X)$ suit la même loi que $X$, ainsi que $Y$.
Soit $\varphi$ une fonction continue bornée,
\[\mathbb{E}(\varphi(X-Y))=\mathbb{E}(\varphi(F_{0,n}(X)-F_{0,n}(Y)))\]
$\varphi(F_{0,n}(X)-F_{0,n}(Y))$ converge en probabilité(d'après la question $T2$), donc en loi vers $0$.
D'où:
\[\mathbb{E}(\varphi(X-Y))=\lim \limits_{n \rightarrow +\infty}\mathbb{E}(\varphi(F_{0,n}(X)-F_{0,n}(Y)))=0\]
Donc $X$ et $Y$ suivent la même loi, i.e. $\pi = \nu$.
Ainsi il existe au plus une seulement probabilité invariante.

\paragraph{T4.}
Soient $x,y \in \E$, on peut montrer par récurrence (de même que dans la question $T2.$) que:
\[\forall l\geq 0, \P({T_+}^{x,y}>n_0l)=\P(F_{0,n_0l}(x)\ne F_{0,n_0l}(y))\leq (1-\epsilon)^l\]
De plus, $(P({T_+}^{x,y}>n)_{n \in \mathbb{N}}$ est une suite décroissante, on a:
\[\forall l, \forall n, n_0l\leq n \leq n_0(l+1) \Rightarrow P({T_+}^{x,y}>n) \leq (1-\epsilon)^l \]
On en déduit que:
\begin{equation*}
\begin{split}
\mathbb{E}[T_+] &=max_{x,y \in \E} \mathbb{E}[T_+^{x,y}] \\
&\leq \binom{\# \E}{2} n_0 \sum_{l\geq 0} (1-\epsilon)^l \\
&\leq \binom{\# \E}{2} \frac{n_0}{\epsilon}
\end{split}
\end{equation*}
\underline{Ainsi, $\mathbb{E}[T_+]$ est borné par $\binom{\# \E}{2} \frac{n_0}{\epsilon}$.}



\end{document}