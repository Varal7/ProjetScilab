\documentclass[a4paper,11pt]{article}
\usepackage[utf8]{inputenc}
\usepackage[T1]{fontenc}
\usepackage[french]{babel}
\usepackage{textcomp}
\usepackage{amsmath,amssymb}
\usepackage[coverpage,fancysections]{polytechnique}

\def \P{\mathbb{P}}
\def \E{\mathcal{E}}

\title{Retour vers le futur}
\author{Sun Qi et Victor Quach}
\date{}

\begin{document}
\maketitle

\subsection*{T1}

\begin{itemize}
\item[\textbullet]
Soit $k \in \E^{bin}=\{0, \dots, M\}$. 

Compte-tenu, de la définition de $F^{bin}$, on a \underline{pour $k<M$} :


\begin{center}
\fbox{\begin{minipage}{0.5\textwidth}
\[P^{bin}(k,k+1) = p\frac{M-k}{M}\]
\end{minipage}}
\end{center}

Notons que le membre de droite est nul pour $k=M$.\\


De même, \underline{pour $k>0$} : 
\begin{center}
\fbox{\begin{minipage}{0.5\textwidth}
\[P^{bin}(k,k-1) = (1-p)\frac{k}{M}\]
\end{minipage}}
\end{center}
Le membre de droite est nul pour $k=0$.\\


Il vient donc, \underline{pour $k \in \E^{bin}$},
\begin{center}
\fbox{\begin{minipage}{0.5\textwidth}
\[P^{bin}(k,k) = 1 - p +\frac{2kp}{M}+\frac{k}{M}\]
\end{minipage}}
\end{center}

\vspace{5mm}
Enfin, \underline{pour $x,y \in \E^{bin}$, si $|x-y|>1$},
\begin{center}
\fbox{\begin{minipage}{0.5\textwidth}
\[P^{bin}(x,y) = 0\]
\end{minipage}}
\end{center}
\vspace{5mm}



\item[\textbullet]
Soit $x,y \in \E^{bin}$.

En utilisant que $\pi^{bin}(x)  = \binom{M}{x}p^x(1-p)^{M-x}$, il vient dans les différents cas :

	\begin{itemize}
		\item 	Si $(x,y) = (k,k+1)$
			\begin{equation}	
			\begin{split}
			\pi^{bin}(k)P(k,k+1) &= \binom{M}{k}p^k(1-p)^{M-k}p\frac{M-k}{M} \\
			&= \frac{M-k}{M}\binom{M}{M-k}p^{k+1}(1-p)^{M-k-1}(1-p)\\
			&= \binom{M-1}{M-k-1}p^{k+1}(1-p)^{M-k-1}(1-p)\\
			&= \binom{M-1}{k}p^{k+1}(1-p)^{M-k-1}(1-p)\\
			&= \frac{k+1}{M}\binom{M}{k+1}p^{k+1}(1-p)^{M-k-1}(1-p)\\
			&= \pi^{bin}(k+1)P(k+1,k) 
			\end{split}
			\end{equation}
		\item   Si $(x,y) = (k,k)$, L'équation est trivialement vérifiée
		\item   Sinon, les deux memmbres sont nuls.
	\end{itemize}
		Finalement, pour tout $x,y \in \E^{bin}$, 
\begin{center}
\fbox{\begin{minipage}{0.5\textwidth}
\[\pi^{bin}(k)P(k,k+1) = \pi^{bin}(k+1)P(k+1,k) \]
\end{minipage}}
\end{center}

\end{itemize}
\end{document}

